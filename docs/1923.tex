% Created 2014-11-19 Wed 14:49
\documentclass[11pt]{article}
\usepackage[utf8]{inputenc}
\usepackage[T1]{fontenc}
\usepackage{fixltx2e}
\usepackage{graphicx}
\usepackage{longtable}
\usepackage{float}
\usepackage{wrapfig}
\usepackage{rotating}
\usepackage[normalem]{ulem}
\usepackage{amsmath}
\usepackage{textcomp}
\usepackage{marvosym}
\usepackage{wasysym}
\usepackage{amssymb}
\usepackage{hyperref}
\tolerance=1000
\author{Thomas Donahue}
\date{\today}
\title{1923}
\hypersetup{
  pdfkeywords={},
  pdfsubject={},
  pdfcreator={Emacs 24.3.1 (Org mode 8.2.7c)}}
\begin{document}

\maketitle
\tableofcontents

\section{About}
\label{sec-1}
Magnificent card game.

\subsection{Testimonials}
\label{sec-1-1}
\begin{itemize}
\item "1923 saved my life." - Local man
\item "So asynchronous!" - Carlos
\item "Surprisingly fun." - Holly
\end{itemize}


\section{Rules}
\label{sec-2}
\subsection{Players}
\label{sec-2-1}
2 or more players are required to play

\subsubsection{Decks}
\label{sec-2-1-1}
The number of decks used depend on the number of players. To
calculate how many decks you should be using, here is a handy-dandy
equation:\\

\texttt{number of decks = ceil( number of players / 3 )}\\

On the other hand, if you lack the proper amount of decks, just go
with what you got. 

\subsection{Structure}
\label{sec-2-2}

\subsubsection{Overview}
\label{sec-2-2-1}
1923 is played in terms of Games, Sets and Matches (akin to Tennis).  

\begin{itemize}
\item A \textbf{game} is much like a "hand" in other card games (e.g. Poker), it is
the smallest chunk of the game. There are multiple games per set.
\item A \textbf{set} is complete when the deck has been depleted. There are
multiple sets per match.
\item A \textbf{match} is what comprises the entire length of the game.
\end{itemize}

\subsubsection{Games}
\label{sec-2-2-2}
\begin{enumerate}
\item Winning
\label{sec-2-2-2-1}

Combined card value (blackjack rules; Ace can be 1 or 11) between
19-23. If multiple players are within that range, the player with the
value closest to 23 wins. If no player is within that range, the
player with the value closest to 19 or 23 wins (being closer to 19 or
23 are equal). Winning player keeps all cards from the round
(including tiebreaker cards). 

\begin{enumerate}
\item Breaking Ties
\label{sec-2-2-2-1-1}

Player with the highest average card value wins. If multiple players
have the same average card value, the players will draw a single card
and highest card (as before, based on value, i.e. 10 is the same as a
king) wins (repeat until winner is found). 

If there aren't enough cards to break a tie in that manner, stay tuned
because we haven't figured it out yet. One idea we're kicking around,
if you'd like guidance, if to take all of the cards in your hand and
play war until there is a winner.
\end{enumerate}

\item Gameplay
\label{sec-2-2-2-2}

\begin{enumerate}
\item Every player draws 2 cards and places them face down (without
looking at them). By custom, the cards are to be placed
side-by-side and to the right hand side in front of the player.
\item Players may look at one of their 2 cards.
\item If a player desires to view the other card, they may due so, but
are required to take another card from the deck as a result.
Players can repeat this step as many times as they would like. This
is done asynchronously (i.e., you needn't wait and see what other
players do). By custom, each additional card must be placed to the
left of the original 2 cards (and so on).
\item Each player must lock their left-most card when they are done
drawing cards (i.e., the card they do not know the value of). Once
a player has locked, they cannot unlock. To lock, the player turns
the card sideways -- a card is locked once the player has removed
their hand from the card.
\item Once each player has locked, all players must turnover their cards
(from right to left) and count their value. Winner is determined as
outlined above.
\end{enumerate}
\end{enumerate}

\subsubsection{Sets}
\label{sec-2-2-3}

\begin{enumerate}
\item Completion
\label{sec-2-2-3-1}

Sets are complete when there aren't enough cards left to carry out
another round, the game is over. What does it mean there aren't enough
cards? First, if there enough cards for every player to have at least
2, then there are enough cards. Any remaining cards after every player
has taken 2 are available in a first come-first serve fashion. If
there are only enough cards for 1 card per player, then the gameplay
reverts to war. In all other cases, there aren't enough cards, and the
remaining cards are dead.

\item Winning
\label{sec-2-2-3-2}

The winner is the player with the most cards. If multiple players have
the same number of cards, they share the win. This is important for
the tabulation of points (see below).

\begin{enumerate}
\item Tabulating Points
\label{sec-2-2-3-2-1}

Once the winning player(s) of the set has been found, their card
totals are transferred into point totals, which are added to a running
sum across all sets until the Match has been completed. 

In general, points are simply 1-to-1 mapped to the number of cards the
player won during that set -- with 1 caveat\ldots{} \textbf{\emph{Aces}}! If the winning
player(s) has any Aces in their hand, they receive what is known as
\emph{Ace benefit}, which is an additional point-per-Ace. Conversely, for
every Ace each losing player has in their hand, they receive what is
known as \emph{Ace damage}, which removes a point.
\end{enumerate}
\end{enumerate}

\subsubsection{Matches}
\label{sec-2-2-4}
\begin{enumerate}
\item Completion
\label{sec-2-2-4-1}

A match is complete when a player has reached N points (where N is a
number we haven't decided yet).

\item Winning
\label{sec-2-2-4-2}

The winner of the match the person who reaches N points first. If
multiple players reach N at the completion of a set, then the player
with the highest card total overall wins. 

If multiple players have the same number of cards\ldots{} we don't know
yet. Perhaps revert to war again (though that's unsatisfying).
\end{enumerate}


\section{Creators, Maintainers and Benevolent Overlords for Life}
\label{sec-3}
\begin{itemize}
\item Thomas Donahue
\item Carlos Asmat
\item Cody Canning
\end{itemize}

\subsection{Contributors}
\label{sec-3-1}
\begin{itemize}
\item Lindsay and Alex
\item Holly Morris
\item Russ Nickerson
\end{itemize}
% Emacs 24.3.1 (Org mode 8.2.7c)
\end{document}
