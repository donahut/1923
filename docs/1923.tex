% Created 2014-11-16 Sun 17:18
\documentclass[11pt]{article}
\usepackage[utf8]{inputenc}
\usepackage[T1]{fontenc}
\usepackage{fixltx2e}
\usepackage{graphicx}
\usepackage{longtable}
\usepackage{float}
\usepackage{wrapfig}
\usepackage{soul}
\usepackage{textcomp}
\usepackage{marvosym}
\usepackage{wasysym}
\usepackage{latexsym}
\usepackage{amssymb}
\usepackage{hyperref}
\tolerance=1000
\providecommand{\alert}[1]{\textbf{#1}}

\title{1923}
\author{Thomas Donahue}
\date{\today}
\hypersetup{
  pdfkeywords={},
  pdfsubject={},
  pdfcreator={Emacs Org-mode version 7.9.3f}}

\begin{document}

\maketitle

\setcounter{tocdepth}{3}
\tableofcontents
\vspace*{1cm}
\section{1923}
\label{sec-1}
\subsection{About}
\label{sec-1-1}

Magnificient card game.
\subsubsection{Testimonials}
\label{sec-1-1-1}

``1923 saved my life.'' - Local man
\subsection{Rules}
\label{sec-1-2}
\subsubsection{Players}
\label{sec-1-2-1}

2 or more players are required to play
\begin{itemize}

\item Decks\\
\label{sec-1-2-1-1}%
The number of decks used are dependent on the number of players. To
calcultae how many decks you should be using, here is a handy-dandy
equation:

number of decks = ceil( number of players / 3 )

\end{itemize} % ends low level
\subsubsection{Gameplay}
\label{sec-1-2-2}
\begin{itemize}

\item Drawing First\\
\label{sec-1-2-2-1}%
All players draw 1 card from the deck. Higest card wins (based on
value i.e. 10 is the same as a king). If there is a tie, tied players
draw again. Drawn cards are discarded from the game -- not reshuffled
into the deck.


\item Each Round
\label{sec-1-2-2-2}%

\begin{itemize}

\item Winning\\
\label{sec-1-2-2-2-1}%
Combined card value (blackjack rules; Ace can be 1 or 11) between
19-23. If multiple players are within that range, the player with the
value closest to 23 wins. If no player is within that range, the
player with the value closest to 19 or 23 wins (being closer to 19 or
23 are equal). Winning player keeps all cards from the round
(including tiebreaker cards). 

\begin{itemize}

\item Breaking Ties\\
\label{sec-1-2-2-2-1-1}%
Player with the lowest average card value wins. If multiple players
have the same average card value, the players will draw a single card
and highest card (as before, based on value, i.e. 10 is the same as a
king) wins (repeat until winner is found).

\end{itemize} % ends low level

\item Drawing Cards
\label{sec-1-2-2-2-2}%
\begin{enumerate}
\item Every player draws 2 cards and places them face down (without
   looking at them). The cards should be placed side-by-side and to
   the right hand side of the ``players space.''
\item Asynchronously, each player can draw as many cards as they desire.
   If a player wishes to look at one of their cards, they must draw
   an additional card. Each additional card must be placed to the
   left of the original 2 cards.
\item Each player must lock their left-most card when they are done
   drawing cards. Once a player has locked, they cannot unlock. To
   lock, the player turns the card sideways.
\item Once each player has locked, all players must turnover their cards
   and count their value. Winner is determined as outlined above.
\end{enumerate}


\end{itemize} % ends low level
\end{itemize} % ends low level
\subsubsection{Winning the game}
\label{sec-1-2-3}

When there aren't enough cards left to carry out another round, the
game is over. The winner is the player with the most cards.
\subsection{Creators}
\label{sec-1-3}

\begin{itemize}
\item Thomas Donahue
\item Carlos Asmat
\item Cody Canning
\end{itemize}

\end{document}
