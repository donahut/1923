% Created 2014-12-08 Mon 21:14
\documentclass[11pt]{article}
\usepackage[utf8]{inputenc}
\usepackage[T1]{fontenc}
\usepackage{fixltx2e}
\usepackage{graphicx}
\usepackage{longtable}
\usepackage{float}
\usepackage{wrapfig}
\usepackage{soul}
\usepackage{textcomp}
\usepackage{marvosym}
\usepackage{wasysym}
\usepackage{latexsym}
\usepackage{amssymb}
\usepackage{hyperref}
\tolerance=1000
\usepackage{fullpage}
\usepackage{parskip}
\providecommand{\alert}[1]{\textbf{#1}}

\title{1923}
\author{A magnificent card game}
\date{}
\hypersetup{
  pdfkeywords={},
  pdfsubject={},
  pdfcreator={Emacs Org-mode version 7.9.3f}}

\begin{document}

\maketitle

\setcounter{tocdepth}{3}
\tableofcontents
\vspace*{1cm}


\section{About}
\label{sec-1}

Magnificent card game.
\subsection{Testimonials}
\label{sec-1-1}

\begin{itemize}
\item ``1923 saved my life.'' - Local man
\item ``So asynchronous!'' - Carlos
\item ``Surprisingly fun.'' - Holly
\item ``Why?'' - An Unnamed Game Design Major 
\end{itemize}
  
\section{Rules}
\label{sec-2}


Rules are great, but paramount are \textbf{Customs and Traditions}. Don't be
an idiot, read Section 3 before playing.
\subsection{Players}
\label{sec-2-1}

2 or more players are required to play
\subsubsection{Decks}
\label{sec-2-1-1}

The number of decks used depend on the number of players. To
calculate how many decks you should be using, here is a handy-dandy
equation:\\

\texttt{number of decks = ceil( number of players / 3 )}\\

On the other hand, if you lack the proper amount of decks, just go
with what you got. 
\subsection{Structure}
\label{sec-2-2}
\subsubsection{Overview}
\label{sec-2-2-1}

1923 is played in terms of Hands and Rounds.   

\begin{itemize}
\item A \textbf{hand} is much like a ``hand'' in other card games (e.g. Poker), it is
  the smallest chunk of 1923. There are multiple hands per round.
\item A \textbf{round} is complete when the deck has been depleted. There are
  multiple rounds per game.
\end{itemize}
\subsubsection{Hands}
\label{sec-2-2-2}
\begin{itemize}

\item Winning a Hand\\
\label{sec-2-2-2-1}%
The goal of a hand is to have a combined card value (blackjack rules; 
Ace can be 1 or 11) between 19-23. If multiple players are within that 
range, the player with the value closest to 23 wins. If no player is 
within that range, the player with the value closest to 19 or 23 wins 
(being closer to 19 or 23 are equal). Winning player keeps all cards 
from the round including the other players hands, tiebreaker cards, etc. 

\begin{itemize}

\item Breaking Ties\\
\label{sec-2-2-2-1-1}%
Ties between players are settled by looking at the highest value card in 
each hand. The player with the highest value card wins the tie. If multiple 
players in the tie have the same highest value card, the second highest card 
per hand is used and so on. 

Should players in a tie have the exact same hand, the game results in a 
``Perfect Tie'' and all tied players recieve a point for that hand. 


\item 3-Ace-23\\
\label{sec-2-2-2-1-2}%
If a player achieves a 23 via 3 Aces (known as a 3-Ace-23), they
automatically win the hand and the round is over. All remaining cards in
draw deck go to the player as part of their round total. In the case that
multiple players achieve a 3-Ace-23, then the tie is resolved just as
outlined above. The winner of the tiebreaker recieves ALL of the
remaining cards.

\end{itemize} % ends low level

\item Gameplay\\
\label{sec-2-2-2-2}%
\begin{enumerate}
\item Every player draws 2 cards and places them face down (without
   looking at them).
\item Players may look at one of their 2 cards.
\item If a player desires to view the other card, they may do so, but
   are required to take another card from the deck as a result.
   Players can repeat this step as many times as they would like. This
   is done asynchronously (i.e., you needn't wait and see what other
   players do).
\item Each player must lock their last card when they are done
   drawing cards (i.e., the card they do not know the value of). Once
   a player has locked, they cannot unlock. To lock, the player turns
   the card sideways -- a card is locked once the player has removed
   their hand from the card.
\item Once each player has locked, all players must turnover their cards
   and count their value. Winner is determined as
   outlined above.
\end{enumerate}

\item Cheating\\
\label{sec-2-2-2-2-3}%
If a player is caught cheating during the play of a hand, they forefeit 
the entire round. They sit out of the play of the rest of the round and 
their cards are out of play until the beginning of the next round. 

\end{itemize} % ends low level
\subsubsection{Rounds}
\label{sec-2-2-3}
\begin{itemize}

\item Completion\\
\label{sec-2-2-3-1}%
A round is complete when there aren't enough cards left to carry out
another. If there enough cards for every player to have at least
2 at the start of the round, then there are enough cards. Any remaining 
cards after every player has taken 2 are available in a first come-first 
serve fashion. In all other cases, there aren't enough cards, and the
remaining cards are dead (i.e. they do not count towards the results of
this round). 


\item Winning\\
\label{sec-2-2-3-2}%
The winner is the player with the most cards. If multiple players have
the same number of cards, the round moves into a tiebreaker. 




\begin{itemize}

\item Breaking Ties\\
\label{sec-2-2-3-2-1}%
If two or more players are tied in number of cards at the end of the 
round, the round goes to the tied player who most recently won a hand.

\end{itemize} % ends low level
\end{itemize} % ends low level
\subsubsection{Games}
\label{sec-2-2-4}


A match is complete when a player has won 3 (or more) rounds, and is at
least 2 rounds ahead of all other players. Thus, for example, if Player A 
wins 3 rounds, but Player B has won 2, then Player A must win another round
(4), in order to win the game.
\section{Customs and Traditions}
\label{sec-3}


1923 has a rich and colourful tradition that resulted in the general acceptance
of some peculiar and distinctive practices. Although from a practical standpoint
they may seem unavailing to the game, these conventions are cherished and
respected by most players. Ignoring these customs is a sign of direct disrespect 
to the game's culture and its beloved creators and contributors.
\subsection{Traditional Practices}
\label{sec-3-1}

\begin{enumerate}
\item The cards shall not be shuffled. They may be mixed and the deck can be cut
   but shuffling is frowned upon. Early mathematical analysis of the game made
   it very clear that shuffling does not affect its mechanics.
\item When picking cards from the deck, they are to be placed side-by-side and to
   the right-hand-side in front of the player.
\item When picking additional cards from the deck, they must be placed to the left
   of the original 2 cards. This results in the leftmost card being always
   unknown, adding a deep metaphorical meaning to the game and evoking the
   sinister connotation of the sinistral card.
\item At the end of each hand, the cards should be revealed from right to left.
   This prolongs the suspense for each player, adding to the bold fun we all
   grew to expect from this magnificent game.
\end{enumerate}
\section{Creators, Maintainers and Benevolent Overlords for Life}
\label{sec-4}

\begin{itemize}
\item Thomas Donahue
\item Carlos Asmat
\item Cody Canning
\end{itemize}
\subsection{Contributors}
\label{sec-4-1}

\begin{itemize}
\item Lindsay and Alex
\item Holly Morris
\item Russ Nickerson
\item Stu Ramgolam
\end{itemize}
  

\end{document}
